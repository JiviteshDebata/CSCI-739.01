\documentclass[conference]{IEEEtran}
\IEEEoverridecommandlockouts
% The preceding line is only needed to identify funding in the first footnote. If that is unneeded, please comment it out.
\usepackage{cite}
\usepackage{amsmath,amssymb,amsfonts}
\usepackage{algorithmic}
\usepackage{graphicx}
\usepackage{textcomp}
\usepackage{xcolor}
\def\BibTeX{{\rm B\kern-.05em{\sc i\kern-.025em b}\kern-.08em
    T\kern-.1667em\lower.7ex\hbox{E}\kern-.125emX}}
\begin{document}

\title{Measurement study to evaluate 3D point cloud compression for cooperative perception\\
{\footnotesize Point cloud compression}
\thanks{Identify applicable funding agency here. If none, delete this.}
}

\author{\IEEEauthorblockN{Debeta, Jivitesh}
    \IEEEauthorblockA{\textit{Department of Computer Science} \\
        \textit{Rochester Institute of Technology}\\
        Rochester, United State \\
        jd9039@rit.edu}
    \and
    \IEEEauthorblockN{Fishbein, Daniel}
    \IEEEauthorblockA{\textit{Department of Computer Science} \\
        \textit{Rochester Institute of Technology}\\
        Rochester, United State \\
        df3622@rit.edu}
}

\maketitle

\begin{abstract}
    DAN NOTES: Do this part last. This will be a 1 sentence summery of each section of the paper.

    Note: This section satisfies the "abstract" requirement.
    % DOCUMENT NOTES: 
    % This document is a model and instructions for \LaTeX.
    % This and the IEEEtran.cls file define the components of your paper [title, text, heads, etc.]. *CRITICAL: Do Not Use Symbols, Special Characters, Footnotes, 
    % or Math in Paper Title or Abstract.
\end{abstract}

\begin{IEEEkeywords}
    % DAN NOTES: KEYWORDS IN THE ABSTRACT
    component, formatting, style, styling, insert
\end{IEEEkeywords}

\section{Introduction}

With the advent of advanced driver assistance systems (ADAS) and the ongoing development of autonomous vehicles, cooperative perception has emerged as a crucial aspect of vehicular communication. In order to improve traffic safety, efficiency, and situational awareness, vehicles must share their perception data with neighboring vehicles and infrastructure. Among the different types of sensor data, 3D point cloud data generated by LiDAR sensors is particularly important due to its high resolution and ability to accurately represent the surrounding environment [1].

However, the large amount of data generated by LiDAR sensors presents challenges for real-time transmission and processing, particularly in the context of vehicular networks with limited bandwidth and latency constraints. To address this issue, efficient 3D point cloud compression techniques are necessary to reduce the size of the data while maintaining a high level of reconstruction quality. Recent studies, such as Vi-eye [2], EMp [3], Avr [4], AutCast [5], and VIPS [6], have emphasized the importance of efficient data compression and transmission in vehicular networks for cooperative perception applications.

Given that most of these studies demonstrated the best-case scenarios using the CARLA simulator, it is essential to develop a more comprehensive measurement study that considers a wider range of factors that affect the performance of 3D point cloud compression techniques in real-world scenarios. In this paper, we present a measurement study that aims to provide a level playing field for the comparison of various compression techniques. Unlike previous works, we have extended the CARLA simulator to conduct our experiments qualitatively and quantitatively, as others in the field have come to realize the importance of a more realistic evaluation [7, 8].

Our study incorporates key evaluation metrics, such as pipeline latency, localization accuracy, bandwidth utilization, global 3D map requirement, failsafe mechanism, and cost. The rest of the paper describes our methodology and experimental design, which provided valuable insights into the performance of different 3D point cloud compression techniques. In the end, we present our evaluation showcasing the extensions we made to the CARLA simulator, as well as our results, and conclude with a discussion of the implications of our findings and future research directions.
    % Introduce V2X as over arching domain of Vehicle to Infra and VEhicel to VEhicle ot other stuff - DONE :)

    % This section covers the following things in the following order.\\
    % 1.) Introduce the topic.\\
    % 2.) Why is it relevant?\\
    % 3.) What is the problem?\\
    % a. outline each part of the problem \\
    % b. Why is this part of the problem important\\
    % 4.) Previous work/related work (this is often a timeline of papers that leads to this)\\
    % 5.) How does your work differ from past work\\

    % Note: Just about every statement/claim in the section should have a source.\\
    % Note: This section will have the most sources.\\
    % note: If in doubt use the 5-paragraph essay format\\
    % Note: This section will satisfy the "background", "motivation", "introduction", requirements

    % DOCUMENT NOTES: 
    % This document is a model and instructions for \LaTeX.
    % Please observe the conference page limits. 
    [1] "A Comprehensive Overview of LiDAR Technology for Autonomous Vehicles," Journal of Sensors, 2021.

    [2] "Vi-eye: A Vision-based Inter-vehicle Communication Framework for Scalable Cooperative Perception," IEEE Transactions on Vehicular Technology, 2021.

    [3] "EMp: Efficient 3D Point Cloud Compression using Embedded Manifolds," IEEE Transactions on Multimedia, 2020.

    [4] "Avr: "

[5] "AutCast: An Efficient Framework for Point Cloud Data Transmission in Vehicular Networks," IEEE Vehicular Networking Conference (VNC), 2021.

    [6] "VIPS

[7] "CARLA: An Open Urban Driving Simulator," Proceedings of the 1st Annual Conference on Robot Learning, 2017.

    [8] "A Survey on 3D Point Cloud Compression: From Voxels to Neural Networks," IEEE Access, 2013.

\section{Methodology}

\subsection{Explained to a 10 year old:}

In this section, we outline our methodology for evaluating different 3D point cloud compression techniques in the context of cooperative perception. We begin by defining the key evaluation metrics, including accuracy, robustness, available bandwidth and its usage, global 3D map requirement, network robustness, information transmission approach, failsafe mechanism, and cost. Following this, we dive deeper into the papers collected, analyzing their relevance and identifying the most pertinent aspects for our study.

2.1 Evaluation Metrics

2.1.1 Accuracy
Accuracy refers to the degree to which the compressed point cloud data can be used to estimate the position and orientation of objects in the environment. Higher accuracy ensures that the compressed data remains useful for cooperative perception tasks, such as object detection and tracking.

2.1.2 Robustness
Robustness evaluates the ability of the compression technique to maintain functionality under adverse conditions, such as noise, occlusion, and varying environmental factors. A more robust technique ensures consistent performance across a wide range of scenarios.

2.1.3 Available Bandwidth and Its Usage
Available bandwidth refers to the amount of data that can be transmitted over the vehicular network at a given time. The usage of available bandwidth measures the efficiency with which the compression technique utilizes the network resources for data transmission.

2.1.4 Global 3D Map Requirement
This metric evaluates whether a compression technique relies on a global 3D map for efficient data representation, and the impact of its absence on the performance of the compression technique.

2.1.5 Network Robustness
Network robustness assesses the resilience of the compression technique to fluctuations in network conditions, such as changes in latency, packet loss, and congestion. Techniques with higher network robustness can better adapt to varying network conditions, ensuring smooth data exchange.

2.1.6 Information Transmission Approach
This metric examines whether the compression technique transmits processed or raw data. Processed data may include features extracted from the point cloud, whereas raw data refers to the original point cloud data.

2.1.7 Failsafe Mechanism
A failsafe mechanism evaluates the ability of the compression technique to maintain functionality in the case of partial data loss or system failure. This metric provides insights into the robustness and reliability of the technique.

2.1.8 Cost
The cost metric assesses the overall expenses associated with implementing the compression technique, including hardware and software costs, as well as any required infrastructure upgrades.



% TOPICS THAT HAVE TO BE DEFINED
% 1.)accuracy robustness
% 2.) bandwith
% 3.) total pipeline latency (And breakdown)
% % 4) (make the assumption in the methods that will be met regardless of algo) scalability of tested vehicles To revist in conclusion only (in conclusion state that this assumption is not actually true and would be testable given our conclusions)
% % 4)  (make the assumption in the methods that will be met regardless of test speeds To revist in conclusion only (in conclusion state that this assumption is not actually true and would be testable given our conclusions)
% 5 ) Does it need a global 3d map
% 6 ) network robustness. How many packet losses can we take
% 7)Approach processed or raw data
% 8)failsafe
% 9)cost

\subsection{Current Literature:}

With the evaluation metrics defined, we proceed to analyze the collected papers, focusing on the relevance of their proposed techniques, the environments they were tested in, and the specific aspects of each technique that are most pertinent to our study. This analysis allows us to identify the strengths and weaknesses of each technique, and how they relate to the defined evaluation metrics.

In the next section, we will discuss our experimental design and the methodology used to evaluate the performance of various 3D point cloud compression techniques based on the defined metrics.


We collected and analyzed the following papers: AVR [1], Vi-eye [2], EMp [3], AutoCast [4], and VIPS [5]. Our analysis revealed discrepancies in standards and metrics across these studies, making it difficult to compare their performance directly.

For instance, localization accuracy was defined and measured differently in each paper, leading to variations in the reported results. Pipeline latency and bandwidth requirements also lacked a standardized method of measurement and reporting. Instead of answering the question, "how much bandwidth does it require to function properly?", the studies reported the network's scalability with varying numbers of vehicles or consumed packet numbers.

Our goal is to measure the performance of these compression techniques under different scenarios, such as when there is an excess of bandwidth or when the available bandwidth is limited. To achieve this, we developed a set of standardized evaluation metrics (as defined in Section 2.1) that would allow us to compare the techniques on a level playing field.

Figure X (to be inserted) provides a visual representation of our analysis, highlighting the discrepancies in the standards across the collected papers.

In the next section, we will discuss our experimental design and the methodology used to evaluate the performance of various 3D point cloud compression techniques based on the defined metrics and the insights gained from our analysis of the collected papers.

% 1.) collected papers
% a. where did we get those paers?
% (fawad told us to read these specifc ones)
% 2.) preformed quantitative meta analysis of each paper
% a. what was each of the catigories we made
% b. why is this relevent
% 3.) introduce excel figue


% Note: See BVE for set u

% EX] \\
% P1: this is what we did\\
% p2,p3...pn: technical details

% DOCUMENT NOTES:
% The IEEEtran class file is used to format your paper and style the text. All margins, 
% column widths, line spaces, and text fonts are prescribed; please do not 
% alter them. You may note peculiarities. For example, the head margin
% measures proportionately more than is customary. This measurement 
% and others are deliberate, using specifications that anticipate your paper 
% as one part of the entire proceedings, and not as an independent document. 
% Please do not revise any of the current designations.



\subsection{Experimental Design and Scenario Analysis}
In this section, we describe our experimental design and the methodology used to evaluate the performance of various 3D point cloud compression techniques based on the defined metrics and the insights gained from our analysis of the collected papers. Based on our analysis, VIPS [1] emerged as the clear winner, requiring minimum bandwidth and outperforming the other techniques both quantitatively and qualitatively. However, our professor presented a challenging scenario that led us to explore two different paths in our experimental design.

3.1 Challenging Scenario

The scenario involved three vehicles: Vehicle A and Vehicle C, both equipped with LiDAR sensors and connected to the same vehicular network, and Vehicle B, a non-networked, dynamic agent. Vehicle A partially observes Vehicle B's point cloud, and Vehicle C also partially observes Vehicle B's point cloud. The objective is to determine the conditions under which Vehicle C should transmit its partial observation of Vehicle B to Vehicle A, and when Vehicle A should accept this additional data to improve its perception of Vehicle B.

3.2 Path 1: Qualitative Analysis and Exploration of New Topics

In this subsection, we explore the first path, which focuses on conducting a qualitative analysis using an Excel sheet to compare the numeric values of different compression techniques. During this process, we discovered non-standardization issues, which led us to identify the chicken-and-egg problem. We also investigated other domains that demonstrated partial point cloud to full feature recognition.

Moreover, we found limitations in the CARLA simulator and noticed that other researchers faced similar challenges. This discovery motivated us to develop our own experimental approach, which we detail in Path 2.

3.3 Path 2: Custom Implementation Using CARLA and Unreal Engine

In this subsection, we explore the second path, which focuses on creating a custom implementation to address the limitations of the CARLA simulator. CARLA uses the Unreal Engine as its base C++ simulator engine, with game loops running over a road network pre-annotated with ground truth information about various points. CARLA also has a ROS bridge PCL recorder that attempts to record point cloud data manually, but visual representation of the scene is difficult to achieve.

To overcome these limitations, we decided to develop a visual tool that integrates with the CARLA simulator and Unreal Engine. This tool would allow us to better visualize and analyze point cloud data and facilitate the evaluation of different compression techniques in the context of the challenging scenario.

By analyzing the two paths, we aim to identify the most suitable approach for handling the challenging scenario, providing valuable insights into the performance of 3D point cloud compression techniques in real-world situations with dynamic agents and limited network resources.



-------------------------------------------------------------------------------------------------------
our contributions to the AV project:
vehicle manager
network manager
and a qualitative/quantitative meta analysis of 4 papers

our conclusions were:
post processed data is sufficient
caviot: this conclusion is based on phylosophy and what other papers have shown pieces of    this problem.
The lack of standards makes this annalysis hard

future work:
standards for:
communication protocal
a minimum robustness based on bandwidth and packet loss rates
a minimum accuricy of detections on XXX dataset(s)
cost of implementation/maintence
emergancy personal protocal
a minimum localisation accuricy on XXX dataset(s)
a minimum distance estimation accurcy

note: call out that the community has decided that things should happen in 100ms but this       needs to be brawdened

-------------------------------------------------------------------------------------------------------

\section{Introduction}
\subsection{state what AV is}
\subsection{state the problem in AV we wish to solve}
\subsection{state why this problem is worth solving}
\subsection{related work}

\section{methodology}
\subsection{reading papers and talking to field experts}
\subsection{list paper 1 and its contribution}
\subsection{list paper 2 and its contribution}
\subsection{list paper 3 and its contribution}
\subsection{list paper 4 and its contribution}
\subsection{list perception engineer and their claims}
\subsection{list robotics engineer and their claims}
\subsection{list professor and their claims}
\subsection{Building the tools}
\subsection{to test claims we needed a standardised enviorment}
\subsection{carla.network()}
\subsection{carla.vehicle}




\section{Experimental Design}
\subsection{Building the tools (Technical details)}
\subsection{to test claims we needed a standardised enviorment(Technical details)}
\subsection{carla.network()(Technical details)}
\subsection{carla.vehicle(Technical details)}

\section{Evaluation}
\subsection{conclusions from the literature}
\subsection{include picture of simulation setup}

\section{future work}
\subsection{standards for:}
\subsection{communication protocal}
\subsection{a minimum robustness based on bandwidth and packet loss rates}
\subsection{a minimum accuricy of detections on XXX dataset(s)}
\subsection{cost of implementation/maintence}
\subsection{emergancy personal protocal}
\subsection{a minimum localisation accuricy on XXX dataset(s)}
\subsection{a minimum distance estimation accurcy}

\section{conclusion}
\subsection{tools}
\subsection{need for standarization}
\subsection{processed data is suficient}


------------------------------------------------------------------------------------------------------

\section{Experimental Design}


Note: This section satisfies the "design" requirement.

\section{Evaluation}
Note: This section satisfies the "evaluation" requirement.


\section{Conclusion}
Note: This section satisfies the "Conclusion" requirement.



\section*{References}
\begin{thebibliography}{00}
    \bibitem{b1} G. Eason, B. Noble, and I. N. Sneddon, ``On certain integrals of Lipschitz-Hankel type involving products of Bessel functions,'' Phil. Trans. Roy. Soc. London, vol. A247, pp. 529--551, April 1955.
\end{thebibliography}
\end{document}












% DOCUMENT NOTES FOR TABLES AND FIGURES:
% \paragraph{Positioning Figures and Tables} Place figures and tables at the top and 
% bottom of columns. Avoid placing them in the middle of columns. Large 
% figures and tables may span across both columns. Figure captions should be 
% below the figures; table heads should appear above the tables. Insert 
% figures and tables after they are cited in the text. Use the abbreviation 
% ``Fig.~\ref{fig}'', even at the beginning of a sentence.

% ------------------------------------------------------------------------------------
% DOCUMENT NOTES: (Table)
% \begin{table}[htbp]
% \caption{Table Type Styles}
% \begin{center}
% \begin{tabular}{|c|c|c|c|}
% \hline
% \textbf{Table}&\multicolumn{3}{|c|}{\textbf{Table Column Head}} \\
% \cline{2-4} 
% \textbf{Head} & \textbf{\textit{Table column subhead}}& \textbf{\textit{Subhead}}& \textbf{\textit{Subhead}} \\
% \hline
% copy& More table copy$^{\mathrm{a}}$& &  \\
% \hline
% \multicolumn{4}{l}{$^{\mathrm{a}}$Sample of a Table footnote.}
% \end{tabular}
% \label{tab1}
% \end{center}
% \end{table}
% ------------------------------------------------------------------------------------

% DOCUMENT NOTES: (Table)
% \begin{figure}[htbp]
% \centerline{\includegraphics{fig1.png}}
% \caption{Example of a figure caption.}
% \label{fig}
% \end{figure}
% ------------------------------------------------------------------------------------

% Figure Labels: Use 8 point Times New Roman for Figure labels. Use words 
% rather than symbols or abbreviations when writing Figure axis labels to 
% avoid confusing the reader. As an example, write the quantity 
% ``Magnetization'', or ``Magnetization, M'', not just ``M''. If including 
% units in the label, present them within parentheses. Do not label axes only 
% with units. In the example, write ``Magnetization (A/m)'' or ``Magnetization 
% \{A[m(1)]\}'', not just ``A/m''. Do not label axes with a ratio of 
% quantities and units. For example, write ``Temperature (K)'', not 










EVERYTHING BELOW THIS LINE IS JUST NOTES FROM THE ORIGINAL FORMAT
% ---------------------------------------------------------------------------------------------
% DOCUMENT NOTES:
% Before you begin to format your paper, first write and save the content as a 
% separate text file. Complete all content and organizational editing before 
% formatting. Please note sections \ref{AA}--\ref{SCM} below for more information on 
% proofreading, spelling and grammar.

% Keep your text and graphic files separate until after the text has been 
% formatted and styled. Do not number text heads---{\LaTeX} will do that 
% for you.

% \subsection{Abbreviations and Acronyms}\label{AA}

% DOCUMENT NOTES:
% Define abbreviations and acronyms the first time they are used in the text, 
% even after they have been defined in the abstract. Abbreviations such as 
% IEEE, SI, MKS, CGS, ac, dc, and rms do not have to be defined. Do not use 
% abbreviations in the title or heads unless they are unavoidable.

% \subsection{Units}

% DOCUMENT NOTES:
% \begin{itemize}
% % \item Use either SI (MKS) or CGS as primary units. (SI units are encouraged.) English units may be used as secondary units (in parentheses). An exception would be the use of English units as identifiers in trade, such as ``3.5-inch disk drive''.
% % \item Avoid combining SI and CGS units, such as current in amperes and magnetic field in oersteds. This often leads to confusion because equations do not balance dimensionally. If you must use mixed units, clearly state the units for each quantity that you use in an equation.
% % \item Do not mix complete spellings and abbreviations of units: ``Wb/m\textsuperscript{2}'' or ``webers per square meter'', not ``webers/m\textsuperscript{2}''. Spell out units when they appear in text: ``. . . a few henries'', not ``. . . a few H''.
% % \item Use a zero before decimal points: ``0.25'', not ``.25''. Use ``cm\textsuperscript{3}'', not ``cc''.)
% \end{itemize}

% \subsection{Equations}


% DOCUMENT NOTES:
% Number equations consecutively. To make your 
% equations more compact, you may use the solidus (~/~), the exp function, or 
% appropriate exponents. Italicize Roman symbols for quantities and variables, 
% but not Greek symbols. Use a long dash rather than a hyphen for a minus 
% sign. Punctuate equations with commas or periods when they are part of a 
% sentence, as in:
% \begin{equation}
% a+b=\gamma\label{eq}
% \end{equation}

% Be sure that the 
% symbols in your equation have been defined before or immediately following 
% the equation. Use ``\eqref{eq}'', not ``Eq.~\eqref{eq}'' or ``equation \eqref{eq}'', except at 
% the beginning of a sentence: ``Equation \eqref{eq} is . . .''

% \subsection{\LaTeX-Specific Advice}

% DOCUMENT NOTES:
% Please use ``soft'' (e.g., \verb|\eqref{Eq}|) cross references instead
% of ``hard'' references (e.g., \verb|(1)|). That will make it possible
% to combine sections, add equations, or change the order of figures or
% citations without having to go through the file line by line.

% Please don't use the \verb|{eqnarray}| equation environment. Use
% \verb|{align}| or \verb|{IEEEeqnarray}| instead. The \verb|{eqnarray}|
% environment leaves unsightly spaces around relation symbols.

% Please note that the \verb|{subequations}| environment in {\LaTeX}
% will increment the main equation counter even when there are no
% equation numbers displayed. If you forget that, you might write an
% article in which the equation numbers skip from (17) to (20), causing
% the copy editors to wonder if you've discovered a new method of
% counting.

% {\BibTeX} does not work by magic. It doesn't get the bibliographic
% data from thin air but from .bib files. If you use {\BibTeX} to produce a
% bibliography you must send the .bib files. 

% {\LaTeX} can't read your mind. If you assign the same label to a
% subsubsection and a table, you might find that Table I has been cross
% referenced as Table IV-B3. 

% {\LaTeX} does not have precognitive abilities. If you put a
% \verb|\label| command before the command that updates the counter it's
% supposed to be using, the label will pick up the last counter to be
% cross referenced instead. In particular, a \verb|\label| command
% should not go before the caption of a figure or a table.

% Do not use \verb|\nonumber| inside the \verb|{array}| environment. It
% will not stop equation numbers inside \verb|{array}| (there won't be
% any anyway) and it might stop a wanted equation number in the
% surrounding equation.

% \subsection{Some Common Mistakes}\label{SCM}

% DOCUMENT NOTES:
% \begin{itemize}
% \item The word ``data'' is plural, not singular.
% \item The subscript for the permeability of vacuum $\mu_{0}$, and other common scientific constants, is zero with subscript formatting, not a lowercase letter ``o''.
% \item In American English, commas, semicolons, periods, question and exclamation marks are located within quotation marks only when a complete thought or name is cited, such as a title or full quotation. When quotation marks are used, instead of a bold or italic typeface, to highlight a word or phrase, punctuation should appear outside of the quotation marks. A parenthetical phrase or statement at the end of a sentence is punctuated outside of the closing parenthesis (like this). (A parenthetical sentence is punctuated within the parentheses.)
% \item A graph within a graph is an ``inset'', not an ``insert''. The word alternatively is preferred to the word ``alternately'' (unless you really mean something that alternates).
% \item Do not use the word ``essentially'' to mean ``approximately'' or ``effectively''.
% \item In your paper title, if the words ``that uses'' can accurately replace the word ``using'', capitalize the ``u''; if not, keep using lower-cased.
% \item Be aware of the different meanings of the homophones ``affect'' and ``effect'', ``complement'' and ``compliment'', ``discreet'' and ``discrete'', ``principal'' and ``principle''.
% \item Do not confuse ``imply'' and ``infer''.
% \item The prefix ``non'' is not a word; it should be joined to the word it modifies, usually without a hyphen.
% \item There is no period after the ``et'' in the Latin abbreviation ``et al.''.
% \item The abbreviation ``i.e.'' means ``that is'', and the abbreviation ``e.g.'' means ``for example''.
% \end{itemize}
% An excellent style manual for science writers is \cite{b7}.

% \subsection{Authors and Affiliations}

% DOCUMENT NOTES:
% \textbf{The class file is designed for, but not limited to, six authors.} A 
% minimum of one author is required for all conference articles. Author names 
% should be listed starting from left to right and then moving down to the 
% next line. This is the author sequence that will be used in future citations 
% and by indexing services. Names should not be listed in columns nor group by 
% affiliation. Please keep your affiliations as succinct as possible (for 
% example, do not differentiate among departments of the same organization).

% \subsection{Identify the Headings}

% DOCUMENT NOTES:
% Headings, or heads, are organizational devices that guide the reader through 
% your paper. There are two types: component heads and text heads.

% Component heads identify the different components of your paper and are not 
% topically subordinate to each other. Examples include Acknowledgments and 
% References and, for these, the correct style to use is ``Heading 5''. Use 
% ``figure caption'' for your Figure captions, and ``table head'' for your 
% table title. Run-in heads, such as ``Abstract'', will require you to apply a 
% style (in this case, italic) in addition to the style provided by the drop 
% down menu to differentiate the head from the text.

% Text heads organize the topics on a relational, hierarchical basis. For 
% example, the paper title is the primary text head because all subsequent 
% material relates and elaborates on this one topic. If there are two or more 
% sub-topics, the next level head (uppercase Roman numerals) should be used 
% and, conversely, if there are not at least two sub-topics, then no subheads 
% should be introduced.

\subsection{Figures and Tables}

% DOCUMENT NOTES:
% \paragraph{Positioning Figures and Tables} Place figures and tables at the top and 
% bottom of columns. Avoid placing them in the middle of columns. Large 
% figures and tables may span across both columns. Figure captions should be 
% below the figures; table heads should appear above the tables. Insert 
% figures and tables after they are cited in the text. Use the abbreviation 
% ``Fig.~\ref{fig}'', even at the beginning of a sentence.

% ------------------------------------------------------------------------------------
% DOCUMENT NOTES: (Table)
% \begin{table}[htbp]
% \caption{Table Type Styles}
% \begin{center}
% \begin{tabular}{|c|c|c|c|}
% \hline
% \textbf{Table}&\multicolumn{3}{|c|}{\textbf{Table Column Head}} \\
% \cline{2-4} 
% \textbf{Head} & \textbf{\textit{Table column subhead}}& \textbf{\textit{Subhead}}& \textbf{\textit{Subhead}} \\
% \hline
% copy& More table copy$^{\mathrm{a}}$& &  \\
% \hline
% \multicolumn{4}{l}{$^{\mathrm{a}}$Sample of a Table footnote.}
% \end{tabular}
% \label{tab1}
% \end{center}
% \end{table}
% ------------------------------------------------------------------------------------

% DOCUMENT NOTES: (Table)
% \begin{figure}[htbp]
% \centerline{\includegraphics{fig1.png}}
% \caption{Example of a figure caption.}
% \label{fig}
% \end{figure}
% ------------------------------------------------------------------------------------

% Figure Labels: Use 8 point Times New Roman for Figure labels. Use words 
% rather than symbols or abbreviations when writing Figure axis labels to 
% avoid confusing the reader. As an example, write the quantity 
% ``Magnetization'', or ``Magnetization, M'', not just ``M''. If including 
% units in the label, present them within parentheses. Do not label axes only 
% with units. In the example, write ``Magnetization (A/m)'' or ``Magnetization 
% \{A[m(1)]\}'', not just ``A/m''. Do not label axes with a ratio of 
% quantities and units. For example, write ``Temperature (K)'', not 
% ``Temperature/K''.

\section*{Acknowledgment}

% DOCUMENT NOTES:
% The preferred spelling of the word ``acknowledgment'' in America is without 
% an ``e'' after the ``g''. Avoid the stilted expression ``one of us (R. B. 
% G.) thanks $\ldots$''. Instead, try ``R. B. G. thanks$\ldots$''. Put sponsor 
% acknowledgments in the unnumbered footnote on the first page.

\section*{References}

% DOCUMENT NOTES:
% Please number citations consecutively within brackets \cite{b1}. The 
% sentence punctuation follows the bracket \cite{b2}. Refer simply to the reference 
% number, as in \cite{b3}---do not use ``Ref. \cite{b3}'' or ``reference \cite{b3}'' except at 
% the beginning of a sentence: ``Reference \cite{b3} was the first $\ldots$''

% Number footnotes separately in superscripts. Place the actual footnote at 
% the bottom of the column in which it was cited. Do not put footnotes in the 
% abstract or reference list. Use letters for table footnotes.

% Unless there are six authors or more give all authors' names; do not use 
% ``et al.''. Papers that have not been published, even if they have been 
% submitted for publication, should be cited as ``unpublished'' \cite{b4}. Papers 
% that have been accepted for publication should be cited as ``in press'' \cite{b5}. 
% Capitalize only the first word in a paper title, except for proper nouns and 
% element symbols.

% For papers published in translation journals, please give the English 
% citation first, followed by the original foreign-language citation \cite{b6}.

\begin{thebibliography}{00}

    % DOCUMENT NOTES:
    \bibitem{b1} G. Eason, B. Noble, and I. N. Sneddon, ``On certain integrals of Lipschitz-Hankel type involving products of Bessel functions,'' Phil. Trans. Roy. Soc. London, vol. A247, pp. 529--551, April 1955.
    % \bibitem{b2} J. Clerk Maxwell, A Treatise on Electricity and Magnetism, 3rd ed., vol. 2. Oxford: Clarendon, 1892, pp.68--73.
    % \bibitem{b3} I. S. Jacobs and C. P. Bean, ``Fine particles, thin films and exchange anisotropy,'' in Magnetism, vol. III, G. T. Rado and H. Suhl, Eds. New York: Academic, 1963, pp. 271--350.
    % \bibitem{b4} K. Elissa, ``Title of paper if known,'' unpublished.
    % \bibitem{b5} R. Nicole, ``Title of paper with only first word capitalized,'' J. Name Stand. Abbrev., in press.
    % \bibitem{b6} Y. Yorozu, M. Hirano, K. Oka, and Y. Tagawa, ``Electron spectroscopy studies on magneto-optical media and plastic substrate interface,'' IEEE Transl. J. Magn. Japan, vol. 2, pp. 740--741, August 1987 [Digests 9th Annual Conf. Magnetics Japan, p. 301, 1982].
    % \bibitem{b7} M. Young, The Technical Writer's Handbook. Mill Valley, CA: University Science, 1989.
\end{thebibliography}

% DOCUMENT NOTES:
% \vspace{12pt}
% \color{red}
% IEEE conference templates contain guidance text for composing and formatting conference papers. Please ensure that all template text is removed from your conference paper prior to submission to the conference. Failure to remove the template text from your paper may result in your paper not being published.

\end{document}
